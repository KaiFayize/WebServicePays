\begin{center}
\thispagestyle{empty}
\vspace{2cm}
\LARGE{\textbf{ABSTRACT}}\\[1.0cm]
\end{center}
\thispagestyle{empty}
\paragraph{}Le but de ce projet était de mettre en place un référentiel de pays, c'est à dire rendre disponible les différentes informations concernant les pays du monde. Ce référentiel devait être implémenté sous la forme d'un WebService JAVA. Afin de rendre l'information encore plus accessible, un client Php ainsi qu'un client JAVA ont été implémentés.}
\paragraph{}Le WebService en lui même se présente sous la forme d'un JAR à éxecuter. Un tomcat intégré permet le déploiement du service. Il est paramétrable grâce à plusieurs fichiers de configuration.
\paragraph{}Le Client JAVA se présente sous la forme d'un code source à intégrer comportant plusieurs méthodes de test. Il dispose de commentaires et d'une Javadoc complète afin de faciliter l'intégration.
\paragraph{}Le Client Php se présente sous la forme d'un site web comportant différentes pages et dont la principale fonction, sous l'onglet "Utiliser Etamine", permet d'interagir avec le WebService.
\paragraph{}Enfin, une application de gestion permet l'interaction avec la base de données du WebService. Elle dispose d'une fonction de login avec mot de passe encrypté. Une fois authentifié, l'utilisateur peut ajouter, supprimer ou modifier des pays de la base de données suivant son niveau d'authentification.