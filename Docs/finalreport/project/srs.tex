\chapter{Spécification des Systèmes Dynamiques}
\section{Technologies utilisées}
\subsection{Maven}
\paragraph{} Maven est un outil de gestion et d'automatisation de projets developpé par la Fondation Apache Software.\\
C'est un outil developpé pour Java, il permet de créer une application à partir de ses sources et d'informations telles que ses dépendances et modules externes.\\
Maven optimise la création et assure le bon déroulement de la fabrication de l'application.\\
Il utilise le paradigme du Project Object Model qui décrit le projet dans un fichier contenant les dépendances, les modules externes, l'ordre de production, le numéro de version, etc...\\
\paragraph{}
Nous avons utilisé cet outil dans le cadre de ce projet car il nous a permis d'optimiser la production des applications dévelopées.\\
Nous avons géneré des sources et centralisé les dépendances et librairies nécessaires grâce à Maven ce qui a permis d'avoir une vision claire du projet et de l'état d'avancement.\\

\subsection{Spring}
\paragraph{} Spring est un framework libre et open-source de développement Java sous licence Apache.\\
Spring est considéré comme un conteneur léger qui crée et met en relation des objets.\\
Il se base sur trois concepts clés qui sont l'inversion de contrôle,qui donne le contrôle de l'éxécution au framework plutôt qu'à l'application elle-même, la programmation orientée aspect, qui sépare le coeur de l'application des aspects techniques, et une couche d'abstraction.\\ 
\paragraph{}
Nous avons utilisé Spring car c'est un framework très puissant et facilement accessible.\\
Il nous a permis de facilement définir l'infrastructure de nos applications et de gérer tous les objets présents à la création et à l'exécution de ces applications.\\

\subsection{SpringWS}
\paragraph{} Spring Web Services est un produit issu du framework Spring permettant la création de Web services.\\
Ce framework facilite la création et l'utilisation de Web services créés en utilisant le protocole SOAP (voir sous-section SOAP et SoapUI).\\
Comme il est basé sur Spring, Spring-WS crée des web services très flexibles et permet même la mise en place de concepts inspirés de Spring comme l'injection de dépendances.\\
\paragraph{}
Nous avons utilisé Spring-WS pour sa capacité à créer des web services à la fois flexibles et complets mais aussi pour sa facilité d'intégration aux composants Spring déjà présents dans les différents applications.\\
Nous avons ainsi pu réutiliser les différentes configurations et expériences liées à Spring que nous avions déjà pour accélerer la production et la mise en place du web service pays.\\

\subsection{Soap et SoapUI}
\paragraph{} SOAP était l'acronyme de Simple Object Access Protocol, un protocole de Remote Procedure Call (RPC) qui permet d'appeler des méthodes à distances.\\
SOAP permet la transmission de messages entre objets distants à l'aide du protocole HTTP ou SMTP.
Il utilise des métadonnées et est composé de deux parties, une enveloppe contenant des informations sur le message et un modèle de données qui contient les données à transmettre.\\
\paragraph{}
Nous avons utilisé SOAP car c'est un protocole qui est indépendant de la plate-forme et du langage utilisé, ce qui a facilité le développement du web service.\\
\paragraph{}
SoapUI est une application open source qui permet le test de web services utilisant le protocole SOAP.\\
Il permet d'inspecter un web service, de simuler une éxécution et de réaliser des tests fonctionnels.\\
\paragraph{}
Nous avons utilisé SoapUI afin de tester notre web service et de localiser les éventuels problèmes de conception ou d'éxécution.\\

\subsection{JAXB}
\paragraph{} JAXB est l'acronyme de Java Architecture for XML Binding, une interface de programmation Java qui permet la création de classes Java à partir d'un fichier XSD et inversement.\\
Grâce à un mapping entre les types XML et Java, il assure la création des classes, de leurs constructeurs ainsi que leurs mutateurs et accesseurs.\\
\paragraph{}
Nous avons utilisé JAXB car il facilite l'utilisation de XML avec Java grâce à son système de schémas.\\
Le framework Spring se configurant à l'aide de fichiers XML, l'utilisation de JAXB a été encore plus importante afin d'être le plus efficient possible.\\

\subsection{Tomcat}
\paragraph{} Tomcat est un conteneur web (serveur) libre appartenant à la Fondation Apache.\\
En plus d'être un serveur, Tomcat gère les servlets ainsi que les Java Server Pages (JSP).\\
Il a été developpé en Java et est donc indépendant de la plate-forme pour son éxécution.\\
Nous avons utilisé Tomcat car c'est aujourd'hui le serveur HTTP multiplate-forme le plus interessant.\\
De plus Spring intègre directement Tomcat, ce qui a facilité la création et le developpement des applications.\\

\subsection{JDBC}
\paragraph{} JDBC est l'acronyme de Java DataBase Connectivity, une interface de programmation Java permettant la communication entre une application Java et une base de données.\\
Cette API fournit des méthodes pour rechercher et modifier une base de données.\\
Nous avons utilisé JDBC pour la communication entre notre référentiel de pays et notre web service.\\
Sa facilité d'utilisation nous a permis de rapidement mettre en place une liaison sûre entre notre base de données et notre web service.\\

\subsection{JUnit}
\paragraph{}JUnit est un framework de test unitaire pour Java.\\
JUnit utilise un système de TestCase qui contiennent des méthodes de test pour tester une classe et des TestSuite qui permettent d'éxécuter certains TestCase prédéfinis.\\
Il permet ainsi de tester certaines parties précises des application pour détécter d'éventuelles erreurs conceptuelles ou de programmation.\\
Nous avons utilisé JUnit pour tester les différentes classes crées dans l'ensemble des applications developpées.\\
Cela nous a permis d'éviter de trouver des erreurs dans le code vers la fin du développement.\\